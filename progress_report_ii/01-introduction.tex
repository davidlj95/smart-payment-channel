\chapter*{Introduction}
\addcontentsline{toc}{chapter}{Introduction}
\markboth{Introduction}{}
\label{chap:introduction}
%\minitoc

In the present document, the second progress report of the \textit{Bitcoin Payment Channels} project, we'll review the progress and milestones achieved during this stage of the project, to check if the project has been following the schedule, get to some conclusions and reschedule the project if necessary to change the project goals or ways to achieve them.

\section{Project current status summary}
\subsection{Summary of the project status in the last progress report}
In the last progress report, after studying and understanding the underlying concepts behind the Bitcoin technology and the details of how each of them are implemented at a low-level, we developed a Python framework to work with so we could easily create complex transactions by developing puzzle-friendly serializable objects. Despite investing time in the creation of that framework, we thought that a new library, puzzle-friendly oriented, and written for a high level of abstraction language like Python, was needed to develop things faster and would save us time rather than using the current libraries available (although we used them in our framework for critical and hard-to-code features like signatures to don't reinvent the wheel\footnote{As a cryptography enthusiast, I personally tried to understand and reimplement ECDSA signatures source code from Vitalik Buterin's library \cite{pybitcointools:online} in my personal free time, and the reimplementation was eventually used after several tests. However, the only real improvement was a better DER ECDSA signatures codification and the knowledge adquired: a better understanding of the algorithm itself}).

With that framework, we tested all of our components to see if their functionality (basically serialization) matched the Bitcoin protocol using some existing Python libraries \cite{python-bitcoinlib:online, pybitcointools:online} and of course, the Bitcoin Core implementation \cite{bitcoin-core:online}, that represents the implementation used by 5989 nodes (86.49\% of the total network). The latter test we performed was a P2PKH transaction fully generated by our framework\footnote{\url{https://tbtc.blockr.io/tx/info/258fb211724412d6ec6a531973c58233143e6ab355623658adc3164a5c70bd5b}}, that was accepted by the network and mined in block 1095557 of the testnet Bitcoin blockchain. 

After that, we searched for information about Bitcoin unidirectional payment channels and designed a script to open / fund, an unidirectional payment channel, but its implementation and test were not yet performed.

\subsection{Current project status}
We found some mistakes in our script to open the Unidirectional Payment Channel, so we modified it until we got a potentially functional script to create the payment channel. Once we designed that script, we found that we had to add some functionality to our framework: be able to create and spend P2SH transactions, for example with scripts that allow multisignature transactions using \code{OP\_CHECKMULTISIG} and also the use of the \code{OP\_CHECKLOCKTIMEVERIFY} as described in BIP-65 \cite{bip-65:online}.

Problems arrived when trying to spend a P2SH multisignature script transaction once it was funded, because of how the signature was performed (we thought in signatures, the input script is replaced always with the scriptPubKey of the UTXO we are spending, but in case of P2SH, it has to be replaced with the redeemScript, and not the scriptPubKey, that contains it hash along the script standard opcodes). We had to read carefully an article explaining multisig\cite{soroushjp_multisig:online} to find the issue, as the Bitcoin core error and message displayed explaining why our transaction wasn't sent to the network did not clarify anything: just said the signature was wrong.

More problems arrived when dealing with \code{OP\_CHECKLOCKTIMEVERIFY}, another important part of our opening / funding transaction script of the Unidirectional Payment Channel. We had to search how the opcode acted, the serialization format of the number to indicate the time we are locking to, and how was the comparison of that lock time was performed (it required to modify the spending transaction locktime field, as 0, the default value was not valid if you use the lock-time feature). To solve that, we had to ask for support to our project teachers, discuss the BIP65\cite{bip-65:online} implementation, and setup an IDE\cite{bitcoin-ide-lopp:online} with the Bitcoin Core \cite{bitcoin-core:online} C++ code to see exactly how was implemented in the latest version until we succesfully funded / opend our channel\footnote{\url{https://tbtc.blockr.io/tx/info/f730e8b212b1f6e48e6c99c0071dee11353dffae50192a1a3ada6aef13b7c818}}, performed payments, and closed it.

After solved this issue, the project development continued separately, as the Bidirectional Payment Channel development stage started and each of us will implement it individually after studying the white papers describing each of them. I studied the multinode Bidirectional Payment Channel proposed by Christian Decker and Roger Wattenhofer \cite{decker2015fast} and in the time of writing this report, I'm in the stage of development of the opening script for the channel. This means, a 2-weeks delay of the project schedule, mainly due to the extra research performed to create P2SH spending transactions and using \code{OP\_CHECKLOCKTIMEVERIFY} and extra delay due to personal timing issues.